\documentclass[12pt]{article}

\usepackage{graphicx} % 用于插入图片
\usepackage{fancyvrb} % 用于代码高亮显示
\usepackage{xeCJK} % 支持中文
\setCJKmainfont{SimSun} % 设置中文主字体(宋体)
\setCJKmonofont{SimSun} % 设置等宽中文字体

\usepackage{caption} % 控制标题格式
\captionsetup{labelsep=period} % 将冒号改为句号
\renewcommand{\figurename}{图} % 将 Figure 改为 图
\usepackage{amsmath} % 数学公式
\usepackage{float} % 强制浮动体位置
\usepackage{datetime} % 用于自定义日期格式

\title{磁场的测量} % 标题
\author{张福轩}
% 设置中文日期格式
\renewcommand{\today}{\number\year 年 \number\month 月 \number\day 日}

\begin{document}

\maketitle

\section{\normalfont 整理表格}
实验数据处理如图\ref{fig:table_data1}、图\ref{fig:table_data2} 和图\ref{fig:table_data3} 所示。

\begin{figure}[H] % 使用 H 强制图片在当前位置放置
    \centering
    \includegraphics[width=\textwidth]{./figures/表11-1.pdf} 
    \caption{表 11-1}
    \label{fig:table_data1}
\end{figure}
\begin{figure}[H] % 使用 H 强制图片在当前位置放置
    \centering
    \includegraphics[width=\textwidth]{./figures/表11-2.pdf} 
    \caption{表 11-2}
    \label{fig:table_data2}
\end{figure}
\begin{figure}[H] % 使用 H 强制图片在当前位置放置
    \centering
    \includegraphics[width=\textwidth]{./figures/表11-3.pdf} 
    \caption{表 11-3}
    \label{fig:table_data3}
\end{figure}

\section{\normalfont 结果展示}

\subsection{\normalfont 螺线管匝数值 $n$ 和霍尔器件灵敏度 $K_H$}


\begin{align*}
n &= 7143 \, \text{ 匝$/m$} \\
K_H &= 188 \, \text{$V \cdot A^{-1} \cdot T^{-1}$} 
\end{align*}

\subsection{\normalfont 绘制表 3 的 $B\sim X $ 曲线}

曲线如图\ref{fig:table_data3_figure} 所示。

\begin{figure}[H] % 使用 H 强制图片在当前位置放置
    \centering
    \includegraphics[width=\textwidth]{./figures/图.pdf} 
    \caption{表 11-3 的 $B\sim X $ 曲线}
    \label{fig:table_data3_figure}
\end{figure}

\subsection{\normalfont 计算螺线管中心的相对误差 $E_r$}

\begin{align*}
B_0 &= 0.072 \, \text{$KGS$} \\
E_r &= 0.28\%
\end{align*}

这里我使用 Excel 进行计算,计算的中间过程及公式见第\ref{sec:calculation_process} 节。



\section{\normalfont 计算过程}
\label{sec:calculation_process}

图\ref{fig:table_data3} 表 11-3 中第一行计算 $B$ 的公式如下,其余的行之后逐层下拉即可:
\begin{Verbatim}[frame=single, fontsize=\small]
    =F26*10/($I$4*$I$2)
\end{Verbatim}


计算 $B_0$ 的公式如下,前面的系数是真空磁导率$\mu _0$:

\begin{Verbatim}[frame=single, fontsize=\small]
    =4*PI()*0.0000001*I3*I5*10
\end{Verbatim}

计算 $E_r$ 的公式为:
\begin{Verbatim}[frame=single, fontsize=\small]
    =ABS(J26-G34)/J26
\end{Verbatim}

Excel 行列关系及计算结果如图\ref{fig:final_table} 所示:
\begin{figure}[H] % 使用 H 强制图片在当前位置放置
    \centering
    \includegraphics[width=\textwidth]{./figures/佐证材料.pdf} 
    \caption{Excel 行列关系及计算结果}
    \label{fig:final_table}
\end{figure}

\end{document}
