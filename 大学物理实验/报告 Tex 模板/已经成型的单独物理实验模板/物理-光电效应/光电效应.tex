\documentclass[12pt]{article}

\usepackage{graphicx} % 用于插入图片
\usepackage{fancyvrb} % 用于代码高亮显示
\usepackage{xeCJK} % 支持中文
\setCJKmainfont{SimSun} % 设置中文主字体(宋体)
\setCJKmonofont{SimSun} % 设置等宽中文字体

\usepackage{caption} % 控制标题格式
\captionsetup{labelsep=period} % 将冒号改为句号
\renewcommand{\figurename}{图} % 将 Figure 改为 图
\usepackage{amsmath} % 数学公式
\usepackage{float} % 强制浮动体位置
\usepackage{datetime} % 用于自定义日期格式

\title{光电效应} % 标题
\author{张福轩}
% 设置中文日期格式
\renewcommand{\today}{\number\year 年 \number\month 月 \number\day 日}

\begin{document}

\maketitle

\section{\normalfont 整理表格}

\subsection{\normalfont 测量普朗克常量}

实验数据处理如图\ref{fig:table_data1} 所示。

\begin{figure}[H] % 使用 H 强制图片在当前位置放置
    \centering
    \includegraphics[width=\textwidth]{./figures/表29-1.pdf} 
    \caption{$U_S-v$ 关系表}
    \label{fig:table_data1}
\end{figure}

\subsection{\normalfont 测量光电管的光电响应特性}

当加速电压 $U = 45 V$ 时,6 种不同光频率所对应的光电流值如表\ref{fig:table_data2} 所示。

\begin{figure}[H] % 使用 H 强制图片在当前位置放置
    \centering
    \includegraphics[width=\textwidth]{./figures/光电相应特性.pdf} 
    \caption{光电响应特性}
    \label{fig:table_data2}
\end{figure}

\subsection{\normalfont 打印数据}

其余内容见结果展示以及附页的光电效应实验报告。


\section{\normalfont 结果展示}

\subsection{\normalfont 斜率 $K$ 及 $E_r$ 结果}

采用线性回归法计算出 $U_S-v$ 直线的斜率如图 \ref{fig:table_data3} 所示:

\begin{figure}[H] % 使用 H 强制图片在当前位置放置
    \centering
    \includegraphics[width=\textwidth]{./figures/拟合图像.pdf} 
    \caption{拟合图像}
    \label{fig:table_data3}
\end{figure}

计算结果如图 \ref{fig:table_data4} 所示:

\begin{figure}[H] % 使用 H 强制图片在当前位置放置
    \centering
    \includegraphics[width=\textwidth]{./figures/计算结果.pdf} 
    \caption{计算结果}
    \label{fig:table_data4}
\end{figure}

\begin{align*}
    K &= 4.316 \times 10^{-15} \, \text{$ V/Hz$} \\
    E_r &= 4.35\%
    \end{align*}

% \begin{align*}
% n &= 7143 \, \text{ 匝$/m$} \\
% K_H &= 188 \, \text{$V \cdot A^{-1} \cdot T^{-1}$} 
% \end{align*}

\subsection{\normalfont 其余结果见附件的打印数据光电效应实验报告}

这里我使用 Excel 进行计算,计算的中间过程及公式见第\ref{sec:calculation_process} 节。

\section{\normalfont 计算过程}
\label{sec:calculation_process}

% 图\ref{fig:table_data3} 表 11-3 中第一行计算 $B$ 的公式如下,其余的行之后逐层下拉即可:
% \begin{Verbatim}[frame=single, fontsize=\small]
%     =F26*10/($I$4*$I$2)
% \end{Verbatim}


% 计算 $B_0$ 的公式如下,前面的系数是真空磁导率$\mu _0$:

% \begin{Verbatim}[frame=single, fontsize=\small]
%     =4*PI()*0.0000001*I3*I5*10
% \end{Verbatim}

计算 $E_r$ 的公式为:
\begin{Verbatim}[frame=single, fontsize=\small]
    =(B23-E21)/E21
\end{Verbatim}

Excel 行列关系及计算结果如图\ref{fig:final_table} 所示:
\begin{figure}[H] % 使用 H 强制图片在当前位置放置
    \centering
    \includegraphics[width=\textwidth]{./figures/佐证材料.pdf} 
    \caption{Excel 行列关系及计算结果}
    \label{fig:final_table}
\end{figure}

\end{document}
