\documentclass[12pt]{article}

\usepackage{graphicx} % 用于插入图片
\usepackage{fancyvrb} % 用于代码高亮显示
\usepackage{xeCJK} % 支持中文
\setCJKmainfont{SimSun} % 设置中文主字体(宋体)
\setCJKmonofont{SimSun} % 设置等宽中文字体

\usepackage{caption} % 控制标题格式
\captionsetup{labelsep=period} % 将冒号改为句号
\renewcommand{\figurename}{图} % 将 Figure 改为 图
\usepackage{amsmath} % 数学公式
\usepackage{float} % 强制浮动体位置
\usepackage{datetime} % 用于自定义日期格式

\title{分光计实验}% 标题
\author{张福轩}
% 设置中文日期格式
\renewcommand{\today}{\number\year 年 \number\month 月 \number\day 日}

\begin{document}

\maketitle

\section{\normalfont 整理表格及结果展示}
%这里可以放报告要求的整理表格

整理表格如图\ref{fig:table_data1},图\ref{fig:table_data2},图\ref{fig:table_data3} 所示。

\begin{figure}[H] % 使用 H 强制图片在当前位置放置
    \centering
    \includegraphics[width=\textwidth]{./figures/表17-1.pdf} 
    \caption{表17-1}
    \label{fig:table_data1}
\end{figure}
\begin{figure}[H] % 使用 H 强制图片在当前位置放置
    \centering
    \includegraphics[width=\textwidth]{./figures/表17-2.pdf} 
    \caption{表17-2}
    \label{fig:table_data2}
\end{figure}
\begin{figure}[H] % 使用 H 强制图片在当前位置放置
    \centering
    \includegraphics[width=\textwidth]{./figures/表17-3.pdf} 
    \caption{表17-3}
    \label{fig:table_data3}
\end{figure}

\section{\normalfont 计算过程}


这里我使用 Excel 进行计算,同时,计算过程中需要进行单位的转化,这里使用 Python 来实现。

\subsection{\normalfont Python 的单位转化过程}

Python 的代码如下:
\begin{Verbatim}[frame=single, fontsize=\small]
def dms_to_degrees(degrees, minutes):
    return degrees + minutes / 60


def degrees_to_dms(degrees):
    d = int(degrees)
    m = (degrees - d) * 60
    return d, round(m)

while True:
    conversion_type = 
    input("请选择转换类型(1:度分转度,2:度转度分):")

    if conversion_type == "1":
        degrees_d = int(input("请输入度分数据的度部分:"))
        minutes_d = int(input("请输入度分数据的分部分:"))
        result = dms_to_degrees(degrees_d, minutes_d)
        print(f"{degrees_d}度 {minutes_d}分 = {result}度")
    elif conversion_type == "2":
        degrees_input = 
        float(input("请输入以度为单位的数据:"))
        d, m = degrees_to_dms(degrees_input)
        print(f"{degrees_input}度 = {d}度 {m}分")
    else:
        print("无效的选择,请输入 1 或 2。")
        
\end{Verbatim}
\subsection{\normalfont Excel 的计算过程}
将单位转化后就可以进行 Excel 的计算了,Excel 的计算公式如下所示。
\\\\
计算 $\lambda /(10^{-6}m)$ 的公式如下,其余列逐列右拉:

\begin{Verbatim}[frame=single, fontsize=\small]
    =$K$2*SIN(RADIANS(B14))/0.000001
\end{Verbatim}

计算 $\Delta \lambda /(10^{-6}m)$ 的公式为:
\begin{Verbatim}[frame=single, fontsize=\small]
    =ABS(B15-B16)
\end{Verbatim}

计算 $E_r$ 的公式为:
\begin{Verbatim}[frame=single, fontsize=\small]
    =B17/B16
\end{Verbatim}

其余内容按公式在课上计算完毕。
Excel 行列关系及计算结果如图\ref{fig:final_table} 所示:
\begin{figure}[H] % 使用 H 强制图片在当前位置放置
    \centering
    \includegraphics[width=\textwidth]{./figures/佐证材料.pdf} 
    \caption{Excel 行列关系及计算结果}
    \label{fig:final_table}
\end{figure}


\end{document}
