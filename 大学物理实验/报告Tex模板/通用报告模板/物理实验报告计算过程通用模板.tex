\documentclass[12pt]{article}

\usepackage{graphicx} % 用于插入图片
\usepackage{fancyvrb} % 用于代码高亮显示
\usepackage{xeCJK} % 支持中文
\setCJKmainfont{SimSun} % 设置中文主字体(宋体)
\setCJKmonofont{SimSun} % 设置等宽中文字体

\usepackage{caption} % 控制标题格式
\captionsetup{labelsep=period} % 将冒号改为句号
\renewcommand{\figurename}{图} % 将 Figure 改为 图
\usepackage{amsmath} % 数学公式
\usepackage{float} % 强制浮动体位置
\usepackage{placeins} % 引入 placeins 包
\usepackage{datetime} % 用于自定义日期格式

\title{物理实验报告 \LaTeX{} 模板}% 标题
\author{填你的名字}
% 设置中文日期格式
\renewcommand{\today}{\number\year 年 \number\month 月 \number\day 日}

\begin{document}

\maketitle

\section{\normalfont 整理表格}
%这里可以放报告要求的整理表格

整理表格如图\ref{fig:table_data}。

\begin{figure}[H] % 使用 H 强制图片在当前位置放置
    \centering
    \includegraphics[width=\textwidth]{./figures/迈克尔逊.pdf} 
    \caption{实验数据表格}
    \label{fig:table_data}
\end{figure}

\section{\normalfont 结果展示}

%这里可以放实验结果展示,我个人认为先展示结果再展示过程比较好,同时这里也可以插入一些可视化的图表。

计算中间及最终结果如下:

\begin{align*}
\overline{\lambda} &= 6.304 \times 10^{-7} \, \text{m} \\
S &= 6.324 \times 10^{-3} \, \text{mm} \\
\Delta \lambda &= 4.216 \times 10^{-8} \, \text{m} \\
\lambda &= (6.304 \pm 0.422) \times 10^{-7} \, \text{m}\\
E_r &= 1.03\%
\end{align*}

这里我使用 python 进行计算,代码运行结果如下:

\begin{Verbatim}[frame=single, fontsize=\small]
    lambda_mean = 6.393e-07 m
    S = 6.324e-03 mm
    Δlambda_mean = 4.216e-08 m
    lambda_mean = 6.393e-07 +/- 4.216e-08 m
    E_r = 1.03%
\end{Verbatim}

\section{\normalfont 计算过程}

% 这里可以放计算的过程,例如 Excel 的计算公式,或 python 的源代码。

完整版的 Python 代码如下:

\begin{Verbatim}[frame=single, fontsize=\small]
    # 已知参数
    lambda_overline = 6.393e-4
    delta_instrument = 0.00005
    delta_d300_mm =
    [0.09610, 0.09590, 0.09620, 0.10587, 0.08589, 0.09542]
    
    # 计算 S
    n = len(delta_d300_mm)

    mean_delta_d300 = sum(delta_d300_mm) / n

    S = 
    (sum((mean_delta_d300 - d) ** 2 
    for d in delta_d300_mm) / (n - 1))** 0.5
    
    # 计算 ΔB
    delta_B = (2 * delta_instrument**2) ** 0.5
    
    # 计算 Δlambda
    delta_lambda = (2 / 300) * (S**2 + delta_B**2) ** 0.5
    
    # 计算 lambda, 单位转换为米
    lambda_overline *= 1e-3  # 转换为 m
    delta_lambda *= 1e-3  # 转换为 m
    
    # lambda_标
    lambda_standard = 6.328e-7  # 单位 m
    
    # 计算 Er
    E_r = 
    abs(lambda_standard - lambda_overline) 
    / lambda_standard * 100
    
    # 输出结果
    print(f"lambda_mean = {lambda_overline:.3e} m")
    print(f"S = {S:.3e} mm")
    print(f"Δlambda = {delta_lambda:.3e} m")
    print(f"lambda = {lambda_overline:.3e}
     +/- {delta_lambda:.3e} m")
    print(f"E_r = {E_r:.2f}%")
    
    
\end{Verbatim}

\end{document}
