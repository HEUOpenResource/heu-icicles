\documentclass[12pt]{article}

\usepackage{graphicx} % 用于插入图片
\usepackage{fancyvrb} % 用于代码高亮显示
\usepackage{xeCJK} % 支持中文
\setCJKmainfont{SimSun} % 设置中文主字体(宋体)
\setCJKmonofont{SimSun} % 设置等宽中文字体

\usepackage{caption} % 控制标题格式
\captionsetup{labelsep=period} % 将冒号改为句号
\renewcommand{\figurename}{图} % 将 Figure 改为 图
\renewcommand{\tablename}{表} % 将 table 改为 表
\usepackage{amsmath} % 数学公式
\usepackage{float} % 强制浮动体位置
\usepackage{placeins} % 引入 placeins 包
\usepackage{datetime} % 用于自定义日期格式

\title{太阳能电池特性的测量} % 标题
\author{张福轩}
% 设置中文日期格式
\renewcommand{\today}{\number\year 年 \number\month 月 \number\day 日}

\begin{document}

\maketitle

\section{\normalfont 整理表格}

\subsection{\normalfont 测量太阳能电池的端电压 $U$ 和通过负载电阻的电流 $I$}

实验数据处理如表\ref{fig:table_data1} 所示。

\begin{table}[H] % 使用 H 强制图片在当前位置放置
    \centering
    \includegraphics[width=0.78\textwidth]{./figures/表1.pdf} 
    \caption{测量太阳能电池的端电压 $U$  和通过负载电阻的电流 $I$}
    \label{fig:table_data1}
\end{table}

% 防止浮动超过当前标题
\FloatBarrier

\section{\normalfont 结果展示}

\subsection{\normalfont 根据表 26-1 测量的 $U$ 和 $I$ 值计算得到的 $P$ 和 $R$ 值}

计算得到的 $P$ 和 $R$ 值如表\ref{fig:table_data2} 所示。

\begin{table}[H] % 使用 H 强制图片在当前位置放置
    \centering
    \includegraphics[width=0.78\textwidth]{./figures/表2.pdf} 
    \caption{根据表 1 测量的 $U$ 和 $I$ 值计算得到的 $P$ 和 $R$ 值}
    \label{fig:table_data2}
\end{table}

\subsection{\normalfont $U-I$ 曲线}

根据表 \ref{fig:table_data1} 测量的 $U$ 和 $I$ 值绘制得到的 $U-I$ 曲线如图 \ref{fig:pict_data1} 所示。

\begin{figure}[H] % 使用 H 强制图片在当前位置放置
    \centering
    \includegraphics[width=0.65\textwidth]{./figures/图1.pdf} 
    \caption{$U-I$ 曲线}
    \label{fig:pict_data1}
\end{figure}

\subsection{\normalfont $P-R$ 特性曲线}

根据表 \ref{fig:table_data2} 测量的 $P$ 和 $R$ 值绘制得到的 $P-R$ 曲线如图 \ref{fig:pict_data2} 所示。

\begin{figure}[H] % 使用 H 强制图片在当前位置放置
    \centering
    \includegraphics[width=0.65\textwidth]{./figures/图2.pdf} 
    \caption{$P-R$ 特性曲线}
    \label{fig:pict_data2}
\end{figure}

\subsection{\normalfont 对应于最大功率的负载电压值 $R_{max}$ 和内阻值 $R_i$}

对应于最大功率的负载电压值 $R_{max}$ 和根据书中式(26-2)计算出的内阻值 $R_i$如表 \ref{fig:table_data3} 所示:

\begin{table}[H] % 使用 H 强制图片在当前位置放置
    \centering
    \includegraphics[width=0.8\textwidth]{./figures/表3.pdf} 
    \caption{最大功率的负载电压值 $R_{max}$ 和内阻值 $R_i$}
    \label{fig:table_data3}
\end{table}

\subsection{\normalfont 最大功率 $P_{max}$ 和开路电压与短路电流的乘积}

最大功率 $P_{max}$ 和开路电压与短路电流的乘积如表 \ref{fig:table_data4} 所示:

\begin{table}[H] % 使用 H 强制图片在当前位置放置
    \centering
    \includegraphics[width=0.8\textwidth]{./figures/表4.pdf} 
    \caption{最大功率 $P_{max}$ 和开路电压与短路电流的乘积}
    \label{fig:table_data4}
\end{table}

\begin{align*}
    \overline F  = 0.783 \\
\end{align*}

这里我使用 Excel 进行计算,计算的中间过程及公式见第\ref{sec:calculation_process} 节。

% 防止浮动超过当前标题
\FloatBarrier

\section{\normalfont 计算过程}
\label{sec:calculation_process}

根据表 \ref{fig:table_data1} 测量的 $U$ 和 $I$ 值计算得到的 $P$ 和 $R$ 值的公式如下。
\newline \newline 电阻 $R$ 的公式:
\begin{Verbatim}[frame=single, fontsize=\small]
    =B5*1000/A5
\end{Verbatim}
功率 $P$ 的公式:
\begin{Verbatim}[frame=single, fontsize=\small]
    =A5*B5
\end{Verbatim}
之后的行数逐层下拉与右拉即可。
\subsection{\normalfont 计算 $R_{max}$ 和 $R_i$}

根据表 \ref{fig:table_data2} 测量的 $P$ 和 $R$ 值计算得到的 $R_{max}$ 和 $R_i$ 的公式如下。
\newline \newline $R_{max}$ 的公式:
\begin{Verbatim}[frame=single, fontsize=\small]
    =INDEX(K3:K22, MATCH(MAX(L3:L22), L3:L22, 0))
\end{Verbatim}
$R_i$ 的公式:
\begin{Verbatim}[frame=single, fontsize=\small]
    =B3*1000/A3
\end{Verbatim}
$\dfrac{R_{max}}{R_i}$ 的公式:
\begin{Verbatim}[frame=single, fontsize=\small]
    =B27/B28
\end{Verbatim}


\subsection{\normalfont 计算填充因数 $F$ 的平均值}

根据表 \ref{fig:table_data2} 测量的 $P$ 和 $R$ 值计算得到的 $P_{max}$ 和 $U_{o}I_{s}$ 的公式如下。
\newline \newline $P_{max}$ 的公式:
\begin{Verbatim}[frame=single, fontsize=\small]
    =MAX(L3:L22)
\end{Verbatim}
$U_{o}I_{s}$ 的公式:
\begin{Verbatim}[frame=single, fontsize=\small]
    =A$3*B$3
\end{Verbatim}
$F=\dfrac{P_{max}}{U_{o}I_{s}}$ 的公式:
\begin{Verbatim}[frame=single, fontsize=\small]
    =H27/H28
\end{Verbatim}
$\overline F$ 的公式:
\begin{Verbatim}[frame=single, fontsize=\small]
    =AVERAGE(H29:K29)
\end{Verbatim}



Excel 行列关系及计算结果如图\ref{fig:final_table} 所示:
\begin{figure}[H] % 使用 H 强制图片在当前位置放置
    \centering
    \includegraphics[width=\textwidth]{./figures/佐证材料.pdf} 
    \caption{Excel 行列关系及计算结果}
    \label{fig:final_table}
\end{figure}

\end{document}
